% Vajame tabelit ja joonist

\section{Varjatud Markovi ahel} 

\subsection{Markovi ahel}

Juhuslike suuruste jada $\{Y_t\}_{t\geq1}$, mis võtab väärtusi hulgal S, nimetatakse diskreetseks juhuslikuks protsessiks seisundiruumiga S. Käesolevas töös vaa\-ta\-me vaid olukorda, kus S on lõplik. Lihtsuse huvides võime S elemente tähistada ka naturaalarvudega: $S = \{1,2,\dots,K\}$, kus $K = |S|$ on elementide arv.~\citep{raag,bremaud}

\begin{definition}~\citep{raag,bremaud}
Diskreetset juhuslikku protsessi $\{Y_t\}_{t\geq1}$ lõpliku seisundiruumiga S nimetatakse Markovi ahelaks, kui iga $t\geq1$ ning $s_1,\dots,s_{t-1},i,j \in S$ korral

\begin{equation*} \label{eq:markovchain}
P(Y_{t+1}=j | Y_{t}=i, Y_{t-1}=s_{t-1}, \dots, Y_{1}=s_{1}) = P(Y_{t+1}=j | Y_{t}=i).
\end{equation*}

\end{definition}

Tõenäosusi $P(Y_{t+1}=j | Y_{t}=i)$ nimetatakse üleminukutõenäosusteks ole\-kust $i$ olekusse $j$ ja vastavat maatriksit $P_t := (p_t(i,j))$, kus

\begin{equation*}
p_t(i,j) := P(Y_{t+1}=j | Y_{t}=i),
\end{equation*}

üleminekutõenäosuste maatriksiks või üleminekumaatriksiks. Jaotust \\ $p_1(s) := P(Y_1=s)$ nimetatakse algjaotuseks.

Markovi ahela $\{Y_t\}_{t\geq1}$ indeksit $t$ nimetatakse mõnikord ka ajaks või hetkeks, kahe hetke vahet nimetatakse sammudeks või sammude arvuks.~\citep{raag}

\subsection{Varjatud Markovi ahel}

\TODO{Defineeri homogeenne Markovi ahel}
\TODO{Kas töö käsitleb ainult homogeenseid varjatud Markovi ahelaid või mitte? Millisel moel?}

Olgu $\{Y_t\}_{t\geq1}$ Markovi ahel sesundite hulgaga $S = \{1,2,\dots,K\}$, $K>1$ ja algjaotusega $P(Y_1=s)$, $s \in S$. Kuigi arvestame töös ka mittehomogeensete ahelatega, jätame lihtsuse huvides aega tähistava indeksi ära, kui see ei tekita segadust. Seega tähistame kõiki üleminekumaatrikseid $\mathbb{P} = (p_{ij})_{i,j \in S}$.


\begin{definition}[Varjatud Markovi ahel]~\citep{lember10}
	Protsessi $X = \{Y_t\}_{t\geq1}$ nimetatakse varjatud Markovi ahelaks kui kehtib:

	\begin{enumerate}
		\item $\{Y_t\}_{t\geq1}$ korral, juhuslikud suurused $\{X_t\}_{t\geq1}$ on omavahel sõltumatud;
		\item iga $t = 1,2,\dots$, korral on $X_t$ sõltuv juhuslikust protsessist $\{Y_t\}_{t\geq1}$ (ja ajast $t$) ainult läbi $Y_t$.
	\end{enumerate}

	Juhuslike protsesside paarile $(X, Y)$ viidatakse ka kui \textit{varjatud Markovi mudelile}.
\end{definition}

Teisisõnu võib varjatud Markovi ahelat kirjeldada järgneva skeemina:

\setcounter{figure}{0}

\begin{figure}[h]
	\centering
	\begin{tikzpicture}[
		->,
		>=stealth',
		shorten >=1pt,
		auto,
		node distance=2cm,
		main node/.style={inner sep=0cm,circle,minimum width=1cm,fill=white,draw,font=\small}]

	\node[main node] (y1) {\dots};
	
	\node[main node] (y2) [right of=y1] {$Y_{k-1}$};
	\node[main node] (x2) [below of=y2] {$X_{k-1}$};
	
	\node[main node] (y3) [right of=y2] {$Y_{k}$};
	\node[main node] (x3) [below of=y3] {$X_{k}$};
	
	\node[main node] (y4) [right of=y3] {$Y_{k+1}$};
	\node[main node] (x4) [below of=y4] {$X_{k+1}$};
	
	\node[main node] (y5) [right of=y4] {\dots};
	
	\node[rectangle, inner sep=2mm,draw=black!100, fit=(y1) (y2) (y3) (y4) (y5)] {};

	\path
		(y1) edge node {} (y2)
		(y2) edge node {} (y3)
		(y3) edge node {} (y4)
		(y4) edge node {} (y5)
		
		(y2) edge node {} (x2)
		(y3) edge node {} (x3)
		(y4) edge node {} (x4)
	;
\end{tikzpicture}

	\caption{Varjatud Markovi ahela kuju.}
	\label{fig:HMM}
\end{figure}

Joonisel \ref{fig:HMM} kasti sees olev osa on meile üldjuhul vaadeldamatu ja sealt tuleb varjatud Markovi ahelatele ka nimi.
