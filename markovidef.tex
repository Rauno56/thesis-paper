% Vajame tabelit ja joonist

\section{Varjatud Markovi ahel} 

\subsection{Markovi ahel}

Juhuslike suuruste jada $\{Y_t\}_{t\geq1}$, mis võtab väärtusi hulgal S, nimetatakse diskreetseks juhuslikuks protsessiks seisundiruumiga S. Käesolevas töös vaa\-ta\-me vaid olukorda, kus S on lõplik. Lihtsuse huvides võime S elemente tähistada ka naturaalarvudega: $S = \{1,2,\dots,K\}$, kus $K = |S|$ on elementide arv.~\citep{raag,bremaud}

\begin{definition}~\citep{raag,bremaud}
Diskreetset juhuslikku protsessi $\{Y_t\}_{t\geq1}$ lõpliku seisundiruumiga S nimetatakse Markovi ahelaks, kui iga $t\geq1$ ning $s_1,\dots,s_{t-1},i,j \in S$ korral

\begin{equation*} \label{eq:markovchain}
P(Y_{t+1}=j | Y_{t}=i, Y_{t-1}=s_{t-1}, \dots, Y_{1}=s_{1}) = P(Y_{t+1}=j | Y_{t}=i).
\end{equation*}

\end{definition}

Tõenäosusi $P(Y_{t+1}=j | Y_{t}=i)$ nimetatakse üleminukutõenäosusteks ole\-kust $i$ olekusse $j$ ja vastavat maatriksit $P_t := (p_t(i,j))$, kus

\begin{equation*}
p_t(i,j) := P(Y_{t+1}=j | Y_{t}=i),
\end{equation*}

üleminekutõenäosuste maatriksiks või üleminekumaatriksiks. Jaotust \\ $p_1(s) := P(Y_1=s)$ nimetatakse algjaotuseks.

Markovi ahela $\{Y_t\}_{t\geq1}$ indeksit $t$ nimetatakse mõnikord ka ajaks või hetkeks, kahe hetke vahet nimetatakse sammudeks või sammude arvuks.~\citep{raag}

\subsection{Varjatud Markovi ahel}

\TODO{Defineeri homogeenne Markovi ahel}
\TODO{Kas töö käsitleb ainult homogeenseid varjatud Markovi ahelaid või mitte? Millisel moel?}

Olgu $\{Y_t\}_{t\geq1}$ Markovi ahel sesundite hulgaga $S = \{1,2,\dots,K\}$, $K>1$ ja algjaotusega $P(Y_1=s)$, $s \in S$. Kuigi arvestame töös ka mittehomogeensete ahelatega, jätame lihtsuse huvides aega tähistava indeksi ära, kui see ei tekita segadust. Seega tähistame kõiki üleminekumaatrikseid $\mathbb{P} = (p_{ij})_{i,j \in S}$.


\begin{definition}[Varjatud Markovi ahel]~\citep{lember10}
	Protsessi $X = \{Y_t\}_{t\geq1}$ nimetatakse varjatud Markovi ahelaks kui kehtib:

	\begin{enumerate}
		\item $\{Y_t\}_{t\geq1}$ korral, juhuslikud suurused $\{X_t\}_{t\geq1}$ on omavahel sõltumatud;
		\item iga $t = 1,2,\dots$, korral on $X_t$ sõltuv juhuslikust protsessist $\{Y_t\}_{t\geq1}$ (ja ajast $t$) ainult läbi $Y_t$.
	\end{enumerate}

	Juhuslike protsesside paarile $(X, Y)$ viidatakse ka kui \textit{varjatud Markovi mudelile}.
\end{definition}

Varjatud Markovi ahela mõistet võib kujutada ka järgneva skeemiga:

\begin{figure}[h]
	\centering
	\begin{tikzpicture}[
		->,
		>=stealth',
		shorten >=1pt,
		auto,
		node distance=2cm,
		main node/.style={inner sep=0cm,circle,minimum width=1cm,fill=white,draw,font=\small}]

	\node[main node] (y1) {\dots};
	
	\node[main node] (y2) [right of=y1] {$Y_{k-1}$};
	\node[main node] (x2) [below of=y2] {$X_{k-1}$};
	
	\node[main node] (y3) [right of=y2] {$Y_{k}$};
	\node[main node] (x3) [below of=y3] {$X_{k}$};
	
	\node[main node] (y4) [right of=y3] {$Y_{k+1}$};
	\node[main node] (x4) [below of=y4] {$X_{k+1}$};
	
	\node[main node] (y5) [right of=y4] {\dots};
	
	\node[rectangle, inner sep=2mm,draw=black!100, fit=(y1) (y2) (y3) (y4) (y5)] {};

	\path
		(y1) edge node {} (y2)
		(y2) edge node {} (y3)
		(y3) edge node {} (y4)
		(y4) edge node {} (y5)
		
		(y2) edge node {} (x2)
		(y3) edge node {} (x3)
		(y4) edge node {} (x4)
	;
\end{tikzpicture}

	\caption{Varjatud Markovi ahela kuju.}
	\label{fig:HMM}
\end{figure}

Joonisel~\ref{fig:HMM} ristküliku sees olev osa on meile üldjuhul vaadeldamatu ja sealt tuleb varjatud Markovi ahelatele ka nimi.

%LEHE BLACK IPA

\begin{example}[Personaalne robot]
Olgu meile ülesandeks õpetada robotile inimeste emotsioonide mõistmist. Meil ei ole tehnilisi vahendeid, et emotsioone mõõta, küll aga oskame masinale õpetada ära tundma mõnda tegevust, mida tema omanik teha võiks.
Selles näites on roboti omaniku emotsionaalne seisund varjatud Markovi ahela vaadeldamatu osa $\{Y_t\}_{t\geq1}$. Lihtsustame selle kahele seisundile ning määrame ühest teise seisundisse minemise tõenäosused tabelis~\ref{fig:HMM.ex.transition}.

\begin{table}[h]
	\centering

	\begin{tabular}{ r | c c }	
						& kurb & õnnelik \\
		\hline
		kurb    & 0,6  & 0,4     \\
		õnnelik & 0,05 & 0,95    \\

	\end{tabular}

	\caption{Üleminekutõenäosused.}
	\label{fig:HMM.ex.transition}
\end{table}

See tähendab, et mingi aja, näiteks tunni lõikes, jääb omaniku tuju kurvaks tõenäosusega 0,6 eeldusel, et see oli eelmine tund kurb, ning muutub õnnelikuks tõenäosusega 0,4.
Nüüd fikseerime vaadeldava osa $\{X_t\}_{t\geq1}$. Oletame, et meie robot oskab vahet teha nelja tegevuse vahel. Olgu nendeks aja veetmine sotsiaalvõrgustikus(SV), sportimine, päris inimesega suhtlemine(sots.) ja nutmine. Vaatluste tõenäosused toome välja tabelis~\ref{fig:HMM.ex.emission}.

\begin{table}[h]
	\centering

	\begin{tabular}{ r | *{4}{c} }	
						& SV 	& sport & sots. & nutt\\
		\hline
			 kurb & 0,3 & 0,15	 & 0,05 & 0,5 \\
		õnnelik & 0,2	& 0,2   & 0,5	& 0,1 \\
	\end{tabular}

	\caption{Emissioonitõenäosused.}
	\label{fig:HMM.ex.emission}
\end{table}

Paneme tähele, et kuna vaatluste jaotused määrab emotsionaalse seisund summeeruvad just read üheks. Nii ei suuda roboti omanik kurb olles 50\% tõenäosusega teha muud kui nutta, ning õnnelikuna tahab ta keskmiselt poole ajast teistega suhelda.

\subsection{Kolm klassikalist probleemi}


\end{example}
